
\documentclass[11pt]{article}
\usepackage[utf8]{inputenc}
\usepackage{amsfonts,amsmath,amssymb}
\title{\textbf{Rapport de stage bimensuel}}
\author{Clément Bonvoisin\\Pierre Ludmann}

\begin{document}

\maketitle

\section{Définition des objectifs}

Segmentation d'un signal :

Départ, Croisère, Demi-tour, Croisière, Demi-tour, Croisière, Arrêt + Autre
\\\\
Clément partirai sur une vision traitement du signal.
Pierre du coté machine learning.
\section{Remarques en vrac}
Remarquons que les exercices (et donc les données obtenues) ne se font pas n'importe comment ; même en cas d'interruption de l'exercice.
En effet, on peut considèrer que l'examen suit le graphe suivant :

(graphe possibilité, attention au retour (en marchant à la case départ))

(Begin, Walk, Turn, Stop, Other)

Ceci permet de réduire le nombre de possibilité pour le machine learning (non?)
\end{document}
